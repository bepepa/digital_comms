\lecture{Phase and Frequency Tracking Loops}{Lecture 7}

\section{Tracking Frequency and Phase}

\subsection{The Need for Tracking}

\begin{frame}
  \frametitle{The Need for Tracking Phase}
  \begin{itemize}
  \item We have discussed how the frequency and phase offset between
    the incoming signal and the local oscillator can be estimated.
  \item With this estimate in hand, the incoming signal $r[n]$ can be
    corrected
    \[
      \hat{r}[n] = r[n] \cdot \exp(-j (2\pi \hat{\Delta f} n + \hat{\theta})),
    \]
    where $\hat{\Delta f}$ and $\hat{\theta}$ are the estimates
    obtained during initial acquisition.
  \item However, if the estimate for the frequency offset is even
    slightly wrong, the phase of the corrected signal will be
    increasingly wrong.
    \begin{itemize}
    \item If $\bar{\Delta f}$ and $\bar{\theta}$ denote the true
      frequency and phase offset, the phase error is
      \[
        \Delta \phi[n] = 2\pi (\bar{\Delta f} - \hat{\Delta f})n +
        (\bar{\theta} - \hat{\theta}).
      \]
    \end{itemize}
  \item Furthermore, the frequency offset will generaly \emph{not} be
    constant; e.g., Doppler shifts and temerature effects lead to
    fluctuations.
  \item These problems cannot be solved by one-time estimation.
  \end{itemize}
\end{frame}

\frame{
  \frametitle{Tracking Loops}
  \begin{itemize}
  \item The problem of phase drift due to inaccurate
    frequency offset estimatione must be mitigated.
  \item This can be accomplished with a feedback control loop; such
    loops are referred to as \emph{Phase-Locked Loops} (PLL).
  \item The goal of a PLL is to derive a signal that accurately
    compensates phase errors in the incoming signal.
    \begin{itemize}
    \item We want to generate a signal $\phi[n]$ such that
      \[
        \delta \phi[n] = (2\pi \bar{\Delta f} n + \bar{\theta}) -
        \phi[n] = 0.
      \]
    \end{itemize}
  \item \emph{Note:} Classical PLLs were derived for the analog
    frontend; we focus exclusively on digital PLL for use in digital
    baseband processing.
  \end{itemize}
}

\frame{
  \frametitle{A Generic PLL}
  \begin{columns}
    \begin{column}{0.5\textwidth}
      \begin{center}
        \includegraphics[width=\textwidth]{diag_PLL_generic.pdf}
      \end{center}
    \end{column}
    \begin{column}{0.5\textwidth}
      \begin{itemize}
      \item The \textbf{phase detector} measures the phase difference
        between $r[n]$ and $e^{j\phi[n]}$.
      \item The \textbf{loop filter} generates a control signal from
        the phase error.
      \item The \textbf{integrator }accumulates past corrections.
      \end{itemize}
    \end{column}
  \end{columns}
}

\frame{
  \frametitle{Phase Detector}
  \begin{itemize}
  \item The phase detector compares the phase of the receiver sample
    $r[n]$ to the phase $\phi[n]$ generated by the tracking loop.
  \item Specifically, it measures
    \[
      \Delta \phi[n] = \arg r[n] \cdot e^{-j \phi[n]} = \phi_r[n] - \phi[n],
    \]
    where $\arg x$ is the phase of the complex number $x$ and
    $\phi_r[n] = \arg r[n]$.
  \item Note that $\hat{r}[n] = r[n] \cdot e^{-j \phi[n]}$ is the phase-corrected
    observation.
    \begin{itemize}
    \item Thus, $\hat{r}[n]$ is suitable for input to the decision
      device.
    \item Amplitude correction may still be needed before $\hat{r}[n]$
      is used for symbol decisions. 
    \end{itemize}
  \end{itemize}
}

\frame{
  \frametitle{Loop Filter}
  \begin{itemize}
  \item the loop filter turns the phase error into a suitable control
    signal.
    \begin{itemize}
    \item The filter scales and smoothes the phase error signal.
    \end{itemize}
  \item The discrete-time loop filter is characterized by its transfer
    function $H_F(z)$.
    \begin{itemize}
    \item We fill consider a \emph{first-order PLL} with:
      \[
        H_F[z] = \alpha_1,
      \]
      i.e., the loop filter merely scales the error signal.
    \end{itemize}
  \item A \emph{second order PLL} has a transfer function given by
    \[
      H_F(z) = \alpha_1 + \frac{\alpha_2 z^{-1}}{1-z^{-1}},
    \]
    i.e., an integrator is added to the proportional component.
  \end{itemize}
}

\frame{
  \frametitle{Integrator}
  \begin{itemize}
  \item The integrator accumulates the information from all past error
    measurements $ \Delta \phi[n]$.
  \item The integrator is given by the difference equation
    \[
      y[n] = y[n-1] + x[n-1].
    \]
    \begin{itemize}
    \item The delay of the input signal is needed to avoid a
      delay-free loop; such loops cannot be realized.
    \end{itemize}
  \item The transfer function of the integrator is
    \[
      H_I(z) = \frac{z^{-1}}{1-z^{-1}}.
    \]
  \end{itemize}
}

\frame{
  \frametitle{Linearized PLL}
  \begin{columns}
    \begin{column}{0.5\textwidth}
      \begin{center}
        \includegraphics[width=\textwidth]{diag_PLL_linear.pdf}
      \end{center}
    \end{column}
    \begin{column}{0.5\textwidth}
      \begin{itemize}
      \item This is the structure we will analyze.
      \item It is a linear, time invariant system.
        \begin{itemize}
        \item Therefore, we can analyze it in terms of impulse
          response, transfer function, etc.
        \end{itemize}
      \end{itemize}
    \end{column}
  \end{columns}
}

\frame{
  \frametitle{Agenda}
  \begin{itemize}
  \item We will analyze the properties of a first and second order
    PLL.
    \begin{itemize}
    \item What is the transfer function between the phase of the
      received signal $\phi_r[n]$ and $\phi[n]$?
    \item What does that imply about the steady state phase error
      $\Delta \phi[n]$, when
      \begin{itemize}
      \item there is an initial phase error or a sudden phase change,
      \item there is a frequency estimation error, i.e.,
        $\bar{\Delta f} - \hat{\Delta f} \neq 0$?
      \end{itemize}
    \item How does noise in the received signal affect the phase
      tracking signal $\phi[n]$?
    \end{itemize}
  \end{itemize}
}

\subsection{First-Order PLL}

\frame{
  \frametitle{First-Order PLL}
  \begin{itemize}
  \item The first-order PLL uses a lop filter with transfer function
    $H_F(z) = \alpha_1$.
  \item In terms of the z-transform, we have a recursive
    relationship between the input $\phi_r[n]$ and the output
    $\phi[n]$
    \begin{align*}
      \Phi(z) & = \Delta \Phi(z) \cdot H(z) \cdot
                \frac{z^{-1}}{1-z^{-1}} \\
      & = (\Phi_r(z) - \Phi(z)) \cdot \frac{\alpha_1 z^{-1}}{1-z^{-1}}.
    \end{align*}
  \item Therefore, the system transfer function is given by
    \[
      H(z) = \frac{\Phi(z)}{\Phi_r(z)} = \frac{\alpha_1 z^{-1}}{1- (1-\alpha_1)z^{-1}}.
    \]
  \item Note that this system has a pole at $z_1 = 1-\alpha_1$; hence,
    it is stable as long as $0 < \alpha_1 < 1$.
  \end{itemize}
}

\frame{
  \frametitle{equency Response}
  \begin{center}
    \includegraphics[width=.8\textwidth]{plot_freq_resp_order1.pdf}
  \end{center}
  \begin{itemize}
  \item The PLL feedback control systems are all lowpass filters.
  \item The bandwidth of the filter is approximately proportional to
    the gain $\alpha_1$.
  \end{itemize}

}

\frame{
  \frametitle{Impulse response}
  \begin{itemize}
  \item Since the transfer function is
    \[
      H(z) = \frac{\alpha_1 z^{-1}}{1- (1-\alpha_1)z^{-1}}.
    \]
    we can find the impulse response
    \[
      h[n] =
      \begin{cases}
        0 & n < 1\\
        \alpha_1 \cdot (1 - \alpha_1)^{n-1} & n \geq 1.
      \end{cases}
    \]
    \begin{itemize}
    \item The delay of one sample is due to the delay in the
      integrator.
    \item The impulse response decays exponentially; since $(1 -
      \alpha_1)^n = e^{\ln(1-\alpha_1)n}$, it follows that the time
      constant is $-\ln(1-\alpha_1)$ which is approximately equal to
      $\alpha_1$ for small $\alpha_1$.
    \end{itemize}
  \end{itemize}
}

\frame{
  \frametitle{Exercise: Step Response}
  \begin{itemize}
  \item An initial phase estimation error $\Delta \theta$ or a sudden
    phase change, can be modeled by an input
    signal
    \[
      \phi_r[n] = \Delta \theta \cdot u[n].
    \]
    while $\phi[n]=0$ for $n\leq0$.
  \item Of interest is how long it takes until $\phi_r[n]-\phi[n]
    \approx 0$.
  \item Using the fact that the z-transform of the unit step signal $u[n]$ is
    \[
      U(z) = \sum_{n=0}^\infty 1 \cdot z^{-n} = \frac{1}{1-z^{-1}},
    \]
    show that
    \[
      \phi_r[n]-\phi[n] = \Delta \theta \cdot (1 - \alpha_1)^n \cdot u[n].
    \]
    % it follows that the z-transform of the difference $\Delta \phi[n]$
    % between $\phi_r[n]$ and $\phi[n]$ is
    % \[
    %   \Delta \Phi(z) =  \Phi(z) - \Phi_r(z) = H(z) \cdot U(z) - U(z) =
    %   \frac{1-z^{-1}}{1- (1-\alpha_1)z^{-1}} \cdot \frac{\Delta \theta}{1-z^{-1}}.
    % \]
  \end{itemize}
}

\frame{
  \frametitle{Step Response}
  \begin{center}
    \includegraphics[width=.8\textwidth]{plot_step_order1.pdf}
  \end{center}
  \begin{itemize}
  \item The phase error dissipates exponentially.
  \item Larger gains $\alpha_1$ lead to more rapid decay.
  \end{itemize}
}

\frame{
  \frametitle{Exercise: Ramp Response}
  \begin{itemize}
  \item When a frequency error $\Delta f$ is present, the phase of the
    input signal can be modeled as a ramp signal
    \[
      \phi_r[n] = 2\pi \Delta f \cdot n \cdot u[n].
    \]
    Again,  $\phi[n]=0$ for $n\leq0$.
  \item Of interest is the phase error $\phi_r[n]-\phi[n]$.
  \item Using the fact that the z-transform of the ramp signal $n
    \cdot u[n]$ is
    \[
      U(z) = \sum_{n=0}^\infty n \cdot z^{-n} = \frac{z^{-1}}{(1-z^{-1})^2},
    \]
    show that
    \[
      \phi_r[n]-\phi[n] = \frac{2\pi \Delta f}{\alpha_1} \cdot (1 - (1 - \alpha_1)^n) \cdot u[n].
    \]
    % it follows that the z-transform of the difference $\Delta \phi[n]$
    % between $\phi_r[n]$ and $\phi[n]$ is
    % \[
    %   \Delta \Phi(z) =  \Phi(z) - \Phi_r(z) = H(z) \cdot U(z) - U(z) =
    %   \frac{1-z^{-1}}{1- (1-\alpha_1)z^{-1}} \cdot \frac{\Delta \theta}{1-z^{-1}}.
    % \]
  \end{itemize}
}

\frame{
  \frametitle{Ramp Response}
  \begin{center}
    \includegraphics[width=.8\textwidth]{plot_ramp_order1.pdf}
  \end{center}
  \begin{itemize}
  \item The phase error does \textbf{not} got to 0.
  \item In steady state, the residual error is $\frac{2\pi \Delta f}{\alpha_1}$
  \item Larger gains $\alpha_1$ lead to smaller steady-state error.
  \end{itemize}
}

\frame{
  \frametitle{Noise Response}
  \begin{itemize}
  \item Results so far, appear to suggest that the gain $\alpha_1$
    should be as close as possible to $\alpha_1=1$.
  \item Let us consider the effect that the noise has on the received
    signal; assume that the noise affecting the phase $\phi_r[n]$ of the received
    signal is
    \begin{itemize}
    \item zero mean with
    \item variance $\sigma_r^2$ and
    \item independent from sample to sample.
    \end{itemize}
  \item Then, the filtered noise at the output, i.e., the noise
    affecting $\phi[n]$ is also zero mean and has variance
    \[
      \sigma_{\phi}^2 = \sigma_r^2 \sum_{n=0}^{\infty} |h[n]|^2
      =  \sigma_r^2 \frac{\alpha_1^2}{1 - (1-\alpha_1)^2} \approx
       \sigma_r^2 \frac{\alpha_1}{2 - \alpha_1}. 
    \]
  \end{itemize}
}

\frame{
  \frametitle{Noise Response}
   \begin{center}
    \includegraphics[width=.8\textwidth]{plot_noise_order1.pdf}
  \end{center}
  \begin{itemize}
    \item The output noise power increases with gain $\alpha_1$.
    \item Choosing $\alpha_1$ presents a trade-off between speed of
      convergence and steady state phase noise.
  \end{itemize}
}

\subsection{Second-Order PLL}

\frame{
  \frametitle{Second-Order PLL}
  \begin{itemize}
  \item The second-order PLL uses a lop filter with transfer function
    $H_F(z) = \alpha_1 + \frac{\alpha_2 z^{-1}}{1 - z^{-1}}$.
  \item The system transfer function is given by
    \[
      H(z) = \frac{\alpha_1 z^{-1} + (\alpha_2-\alpha_1)z^{-2}}
      {1- (2-\alpha_1)z^{-1} (1-\alpha_1+\alpha_2)z^{-2}}.
    \]
  \item This system has two poles at
    \[
      z_{1,2} = 1-\frac{\alpha_1}{2} \pm \frac{\sqrt{\alpha_1^2 - 4\alpha_2}}{2}.
    \]
     A sufficient condition for stabilityis  $0 < \alpha_2 < \alpha_1 < 1$.
  \end{itemize}
}

\frame{
  \frametitle{Difference Equation}
  \begin{itemize}
  \item While it is possible to find the response of the system in
    response to impulse, step or ramp, it is tedious and the resulting
    expressions do not provide great insight.
    \begin{itemize}
    \item Obviously, the impulse response depends on the pole
      locations $p_{1,2}$ and has the gerneral form
      \[
        h[n] = (A p_1^{n-1} + B p_2^{n-1}) \cdot u[n].
      \]
    \item Fundamental differences exist between purely real poles
      (i.e., $\alpha_1^2 - 4\alpha_2 \geq 0$ and complx poles.
    \item The quantity $\xi = \sqrt{\frac{\alpha_1^2}{4\alpha_2}}$ is called
      the damping factor.
    \end{itemize}
  \item We will compute the response to different inputs and the noise
    response numerically from the difference equation for the system.
  \end{itemize}
}

\frame{
  \frametitle{Frequency Response}
  \begin{center}
    \includegraphics[width=.8\textwidth]{plot_freq_resp_order2.pdf}
  \end{center}
  \begin{itemize}
  \item The PLL feedback control systems are all lowpass filters.
  \item The bandwidth of the filter decreases with damping factor
    $\xi$.
  \end{itemize}
}

\frame{
  \frametitle{Exercise: Difference Equation for Second Order PLL}
  \begin{itemize}
  \item From the system transfer function 
    \[
      H(z) = \frac{\alpha_1 z^{-1} + (\alpha_2-\alpha_1)z^{-2}}
      {1- (2-\alpha_1)z^{-1} (1-\alpha_1+\alpha_2)z^{-2}}.
    \]
    find the difference equation that relates the input phase
    $\phi_r[n]$ to the output phase $\phi[n]$.
  \item<2-> Answer:
    \begin{align*}
      \phi[n] = & \alpha_1 \phi_r[n-1] + (\alpha_2-\alpha_1)
                  \phi_r[n-2] +\\
      & (2-\alpha_1) \phi[n-1] + (1-\alpha_1+\alpha_2) \phi[n-2]
    \end{align*}
  \end{itemize}
}

\frame{
  \frametitle{equency Response}
  \begin{center}
    \includegraphics[width=.8\textwidth]{plot_imp_resp_order2.pdf}
  \end{center}
  \begin{itemize}
  \item The time constant of all filter is similar; it is set by $\alpha_!$
  \item With lower damping factors, the impolse response ``rings''.
    $\xi$.
  \end{itemize}
}


\frame{
  \frametitle{Step Response}
  \begin{center}
    \includegraphics[width=.8\textwidth]{plot_step_order2.pdf}
  \end{center}
  \begin{itemize}
  \item The phase error approaches 0 exponentially for all cases.
  \item Smaller damping factors exhibit some overshoot.
  \end{itemize}
}


\frame{
  \frametitle{Ramp Response}
  \begin{center}
    \includegraphics[width=.8\textwidth]{plot_ramp_order_2.pdf}
  \end{center}
  \begin{itemize}
  \item For second order systems the steady-state error is equal to 0.
  \item Smaller damping factors lead to more rapid decrease in the
    phase error.
  \end{itemize}
}

\frame{
  \frametitle{Noise Response}
   \begin{center}
    \includegraphics[width=.8\textwidth]{plot_noise_order2.pdf}
  \end{center}
  \begin{itemize}
    \item The output noise power decreases with damping factor $\xi$,
      but not by a lot.
    \item Rapid convergence, favors damping factors $\xi < 1$.
    \end{itemize}
}






%%% Local Variables:
%%% mode: latex
%%% TeX-master: t
%%% End:
